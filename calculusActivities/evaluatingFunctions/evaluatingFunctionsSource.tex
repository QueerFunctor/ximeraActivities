\headline{In this activity we practice evaluating functions at numbers and other functions.} % A one sentence description of the activity
\activitytitle{Evaluating Functions} % the title of the activity

We'll start off easy, asking you a set of questions that you can
probably do.

%%%%%%%%%%%%%%%%%%%%%%%%%%%%%%%%%%%%%%%%%%%%%%%%%%%%%%%%%%%%
%%%%%%%%%%%%%%%%%%%%%%%%%%%%%%%%%%%%%%%%%%%%%%%%%%%%%%%%%%%%
%% Problem
%%%%%%%%%%%%%%%%%%%%%%%%%%%%%%%%%%%%%%%%%%%%%%%%%%%%%%%%%%%%
%%%%%%%%%%%%%%%%%%%%%%%%%%%%%%%%%%%%%%%%%%%%%%%%%%%%%%%%%%%%

\begin{shuffle}
\begin{exercise}
Given that $f(x)=x^4+2 x^2+4 x+5$, evaluate $f(0.5)$. Express your answer in decimal notation.
\begin{solution}
\begin{hint}
$f(0.5)=(0.5)^4+2 (0.5)^2+4 (0.5)+5$.
\end{hint}
\begin{hint}
$f(0.5)=7.5625$.
\end{hint}
The value of the function $f(x) = x^4+2 x^2+4 x+5$, evaluated at $x=0.5$, is $\answer{7.5625}$.
\end{solution}
\end{exercise}

\begin{exercise}
Given that $f(x)=x^4+2 x^3-2 x^2-4 x-2$, evaluate $f(-1)$. Express your answer in decimal notation.
\begin{solution}
\begin{hint}
$f(-1)=(-1)^4+2 (-1)^3-2 (-1)^2-4 (-1)-2$.
\end{hint}
\begin{hint}
$f(-1)=-1$.
\end{hint}
The value of the function $f(x) = x^4+2 x^3-2 x^2-4 x-2$, evaluated at $x=-1$, is $\answer{-1}$.
\end{solution}
\end{exercise}

\begin{exercise}
Given that $f(x)=2 x^2+2 x-5$, evaluate $f(-5)$. Express your answer in decimal notation.
\begin{solution}
\begin{hint}
$f(-5)=2 (-5)^2+2 (-5)-5$.
\end{hint}
\begin{hint}
$f(-5)=35$.
\end{hint}
The value of the function $f(x) = 2 x^2+2 x-5$, evaluated at $x=-5$, is $\answer{35}$.
\end{solution}
\end{exercise}

\begin{exercise}
Given that $f(x)=x^4+2 x^2+3 x-2$, evaluate $f(-2.8)$. Express your answer in decimal notation.
\begin{solution}
\begin{hint}
$f(-2.8)=(-2.8)^4+2 (-2.8)^2+3 (-2.8)-2$.
\end{hint}
\begin{hint}
$f(-2.8)=66.7456$.
\end{hint}
The value of the function $f(x) = x^4+2 x^2+3 x-2$, evaluated at $x=-2.8$, is $\answer{66.7456}$.
\end{solution}
\end{exercise}

\begin{exercise}
Given that $f(x)=-x^2-x+4$, evaluate $f(-3.1)$. Express your answer in decimal notation.
\begin{solution}
\begin{hint}
$f(-3.1)=-(-3.1)^2-(-3.1)+4$.
\end{hint}
\begin{hint}
$f(-3.1)=-2.51$.
\end{hint}
The value of the function $f(x) = -x^2-x+4$, evaluated at $x=-3.1$, is $\answer{-2.51}$.
\end{solution}
\end{exercise}

\begin{exercise}
Given that $f(x)=-3 x^2-4 x+4$, evaluate $f(-2)$. Express your answer in decimal notation.
\begin{solution}
\begin{hint}
$f(-2)=-3 (-2)^2-4 (-2)+4$.
\end{hint}
\begin{hint}
$f(-2)=0$.
\end{hint}
The value of the function $f(x) = -3 x^2-4 x+4$, evaluated at $x=-2$, is $\answer{0}$.
\end{solution}
\end{exercise}

\begin{exercise}
Given that $f(x)=-2 x^3+x^2+3$, evaluate $f(4.2)$. Express your answer in decimal notation.
\begin{solution}
\begin{hint}
$f(4.2)=-2 (4.2)^3+(4.2)^2+3$.
\end{hint}
\begin{hint}
$f(4.2)=-127.536$.
\end{hint}
The value of the function $f(x) = -2 x^3+x^2+3$, evaluated at $x=4.2$, is $\answer{-127.536}$.
\end{solution}
\end{exercise}

\begin{exercise}
Given that $f(x)=x^4-2 x^3+x^2-x-4$, evaluate $f(-0.5)$. Express your answer in decimal notation.
\begin{solution}
\begin{hint}
$f(-0.5)=(-0.5)^4-2 (-0.5)^3+(-0.5)^2-(-0.5)-4$.
\end{hint}
\begin{hint}
$f(-0.5)=-2.9375$.
\end{hint}
The value of the function $f(x) = x^4-2 x^3+x^2-x-4$, evaluated at $x=-0.5$, is $\answer{-2.9375}$.
\end{solution}
\end{exercise}

\begin{exercise}
Given that $f(x)=-2 x^3-3 x^2-4 x$, evaluate $f(2.4)$. Express your answer in decimal notation.
\begin{solution}
\begin{hint}
$f(2.4)=-2 (2.4)^3-3 (2.4)^2-4 (2.4)$.
\end{hint}
\begin{hint}
$f(2.4)=-54.528$.
\end{hint}
The value of the function $f(x) = -2 x^3-3 x^2-4 x$, evaluated at $x=2.4$, is $\answer{-54.528}$.
\end{solution}
\end{exercise}

\begin{exercise}
Given that $f(x)=-x^3-x-3$, evaluate $f(-2.9)$. Express your answer in decimal notation.
\begin{solution}
\begin{hint}
$f(-2.9)=-(-2.9)^3-(-2.9)-3$.
\end{hint}
\begin{hint}
$f(-2.9)=24.289$.
\end{hint}
The value of the function $f(x) = -x^3-x-3$, evaluated at $x=-2.9$, is $\answer{24.289}$.
\end{solution}
\end{exercise}

\begin{exercise}
Given that $f(x)=x^4+2 x^2-3 x$, evaluate $f(-1.9)$. Express your answer in decimal notation.
\begin{solution}
\begin{hint}
$f(-1.9)=(-1.9)^4+2 (-1.9)^2-3 (-1.9)$.
\end{hint}
\begin{hint}
$f(-1.9)=25.9521$.
\end{hint}
The value of the function $f(x) = x^4+2 x^2-3 x$, evaluated at $x=-1.9$, is $\answer{25.9521}$.
\end{solution}
\end{exercise}

\begin{exercise}
Given that $f(x)=x^4+x^3-2 x^2-x-2$, evaluate $f(-0.9)$. Express your answer in decimal notation.
\begin{solution}
\begin{hint}
$f(-0.9)=(-0.9)^4+(-0.9)^3-2 (-0.9)^2-(-0.9)-2$.
\end{hint}
\begin{hint}
$f(-0.9)=-2.7929$.
\end{hint}
The value of the function $f(x) = x^4+x^3-2 x^2-x-2$, evaluated at $x=-0.9$, is $\answer{-2.7929}$.
\end{solution}
\end{exercise}

\begin{exercise}
Given that $f(x)=-3 x^2-2 x+1$, evaluate $f(4.2)$. Express your answer in decimal notation.
\begin{solution}
\begin{hint}
$f(4.2)=-3 (4.2)^2-2 (4.2)+1$.
\end{hint}
\begin{hint}
$f(4.2)=-60.32$.
\end{hint}
The value of the function $f(x) = -3 x^2-2 x+1$, evaluated at $x=4.2$, is $\answer{-60.32}$.
\end{solution}
\end{exercise}

\begin{exercise}
Given that $f(x)=x^2-x-2$, evaluate $f(4.1)$. Express your answer in decimal notation.
\begin{solution}
\begin{hint}
$f(4.1)=(4.1)^2-(4.1)-2$.
\end{hint}
\begin{hint}
$f(4.1)=10.71$.
\end{hint}
The value of the function $f(x) = x^2-x-2$, evaluated at $x=4.1$, is $\answer{10.71}$.
\end{solution}
\end{exercise}

\begin{exercise}
Given that $f(x)=-2 x^2-3 x+5$, evaluate $f(-2.2)$. Express your answer in decimal notation.
\begin{solution}
\begin{hint}
$f(-2.2)=-2 (-2.2)^2-3 (-2.2)+5$.
\end{hint}
\begin{hint}
$f(-2.2)=1.92$.
\end{hint}
The value of the function $f(x) = -2 x^2-3 x+5$, evaluated at $x=-2.2$, is $\answer{1.92}$.
\end{solution}
\end{exercise}

\begin{exercise}
Given that $f(x)=-x^2+4 x-3$, evaluate $f(0.1)$. Express your answer in decimal notation.
\begin{solution}
\begin{hint}
$f(0.1)=-(0.1)^2+4 (0.1)-3$.
\end{hint}
\begin{hint}
$f(0.1)=-2.61$.
\end{hint}
The value of the function $f(x) = -x^2+4 x-3$, evaluated at $x=0.1$, is $\answer{-2.61}$.
\end{solution}
\end{exercise}

\begin{exercise}
Given that $f(x)=-x^2-4 x-2$, evaluate $f(-1.1)$. Express your answer in decimal notation.
\begin{solution}
\begin{hint}
$f(-1.1)=-(-1.1)^2-4 (-1.1)-2$.
\end{hint}
\begin{hint}
$f(-1.1)=1.19$.
\end{hint}
The value of the function $f(x) = -x^2-4 x-2$, evaluated at $x=-1.1$, is $\answer{1.19}$.
\end{solution}
\end{exercise}

\begin{exercise}
Given that $f(x)=-x^4-2 x^2-x+2$, evaluate $f(-0.3)$. Express your answer in decimal notation.
\begin{solution}
\begin{hint}
$f(-0.3)=-(-0.3)^4-2 (-0.3)^2-(-0.3)+2$.
\end{hint}
\begin{hint}
$f(-0.3)=2.1119$.
\end{hint}
The value of the function $f(x) = -x^4-2 x^2-x+2$, evaluated at $x=-0.3$, is $\answer{2.1119}$.
\end{solution}
\end{exercise}

\begin{exercise}
Given that $f(x)=5-4 x$, evaluate $f(0.5)$. Express your answer in decimal notation.
\begin{solution}
\begin{hint}
$f(0.5)=5-4 (0.5)$.
\end{hint}
\begin{hint}
$f(0.5)=3$.
\end{hint}
The value of the function $f(x) = 5-4 x$, evaluated at $x=0.5$, is $\answer{3}$.
\end{solution}
\end{exercise}

\begin{exercise}
Given that $f(x)=x^4+x-3$, evaluate $f(2.3)$. Express your answer in decimal notation.
\begin{solution}
\begin{hint}
$f(2.3)=(2.3)^4+(2.3)-3$.
\end{hint}
\begin{hint}
$f(2.3)=27.2841$.
\end{hint}
The value of the function $f(x) = x^4+x-3$, evaluated at $x=2.3$, is $\answer{27.2841}$.
\end{solution}
\end{exercise}
\end{shuffle}


%%%%%%%%%%%%%%%%%%%%%%%%%%%%%%%%%%%%%%%%%%%%%%%%%%%%%%%%%%%%
%%%%%%%%%%%%%%%%%%%%%%%%%%%%%%%%%%%%%%%%%%%%%%%%%%%%%%%%%%%%
%% Problem
%%%%%%%%%%%%%%%%%%%%%%%%%%%%%%%%%%%%%%%%%%%%%%%%%%%%%%%%%%%%
%%%%%%%%%%%%%%%%%%%%%%%%%%%%%%%%%%%%%%%%%%%%%%%%%%%%%%%%%%%%


\begin{shuffle}
\begin{exercise}
Given that $f(x)=-4$, evaluate $f\left(2 \pi\right)$. Express your answer in an exact form.
\begin{solution}
\begin{hint}
$f\left(2 \pi\right)=-4$. Note, you could stop here and have a perfectly acceptable answer. However, you could also recall facts about the unit circle and continue on. 
\end{hint}
\begin{hint}
$f\left(2 \pi\right)=-4$.
\end{hint}
The value of the function $f(x) = -4$, evaluated at $x=2 \pi$, is $\answer{-4}$.
\end{solution}
\end{exercise}

\begin{exercise}
Given that $f(x)=-2 \sin ^3(2 x)-2 \sin (2 x)-2 \cos ^2(2 x)+5$, evaluate $f\left(\frac{5 \pi }{6}\right)$. Express your answer in an exact form.
\begin{solution}
\begin{hint}
$f\left(\frac{5 \pi }{6}\right)=-2 \sin ^3(2 \left(\frac{5 \pi }{6}\right))-2 \sin (2 \left(\frac{5 \pi }{6}\right))-2 \cos ^2(2 \left(\frac{5 \pi }{6}\right))+5$. Note, you could stop here and have a perfectly acceptable answer. However, you could also recall facts about the unit circle and continue on. 
\end{hint}
\begin{hint}
$f\left(\frac{5 \pi }{6}\right)=\frac{9}{2}+\frac{7 \sqrt{3}}{4}$.
\end{hint}
The value of the function $f(x) = -2 \sin ^3(2 x)-2 \sin (2 x)-2 \cos ^2(2 x)+5$, evaluated at $x=\frac{5 \pi }{6}$, is $\answer{\frac{9}{2}+\frac{7 \sqrt{3}}{4}}$.
\end{solution}
\end{exercise}

\begin{exercise}
Given that $f(x)=\sin (2 x)-2 \cos ^2(2 x)-1$, evaluate $f\left(\frac{\pi }{6}\right)$. Express your answer in an exact form.
\begin{solution}
\begin{hint}
$f\left(\frac{\pi }{6}\right)=\sin (2 \left(\frac{\pi }{6}\right))-2 \cos ^2(2 \left(\frac{\pi }{6}\right))-1$. Note, you could stop here and have a perfectly acceptable answer. However, you could also recall facts about the unit circle and continue on. 
\end{hint}
\begin{hint}
$f\left(\frac{\pi }{6}\right)=\frac{\sqrt{3}}{2}-\frac{3}{2}$.
\end{hint}
The value of the function $f(x) = \sin (2 x)-2 \cos ^2(2 x)-1$, evaluated at $x=\frac{\pi }{6}$, is $\answer{\frac{\sqrt{3}}{2}-\frac{3}{2}}$.
\end{solution}
\end{exercise}

\begin{exercise}
Given that $f(x)=2 \cos ^3(x)-2$, evaluate $f\left(\frac{3 \pi }{2}\right)$. Express your answer in an exact form.
\begin{solution}
\begin{hint}
$f\left(\frac{3 \pi }{2}\right)=2 \cos ^3(\left(\frac{3 \pi }{2}\right))-2$. Note, you could stop here and have a perfectly acceptable answer. However, you could also recall facts about the unit circle and continue on. 
\end{hint}
\begin{hint}
$f\left(\frac{3 \pi }{2}\right)=-2$.
\end{hint}
The value of the function $f(x) = 2 \cos ^3(x)-2$, evaluated at $x=\frac{3 \pi }{2}$, is $\answer{-2}$.
\end{solution}
\end{exercise}

\begin{exercise}
Given that $f(x)=-\sin ^2(x)-4 \cos (2 x)+5$, evaluate $f\left(\frac{4 \pi }{3}\right)$. Express your answer in an exact form.
\begin{solution}
\begin{hint}
$f\left(\frac{4 \pi }{3}\right)=-\sin ^2(\left(\frac{4 \pi }{3}\right))-4 \cos (2 \left(\frac{4 \pi }{3}\right))+5$. Note, you could stop here and have a perfectly acceptable answer. However, you could also recall facts about the unit circle and continue on. 
\end{hint}
\begin{hint}
$f\left(\frac{4 \pi }{3}\right)=\frac{25}{4}$.
\end{hint}
The value of the function $f(x) = -\sin ^2(x)-4 \cos (2 x)+5$, evaluated at $x=\frac{4 \pi }{3}$, is $\answer{\frac{25}{4}}$.
\end{solution}
\end{exercise}

\begin{exercise}
Given that $f(x)=\sin ^2(2 x)-\sin (x)-5$, evaluate $f\left(\frac{11 \pi }{6}\right)$. Express your answer in an exact form.
\begin{solution}
\begin{hint}
$f\left(\frac{11 \pi }{6}\right)=\sin ^2(2 \left(\frac{11 \pi }{6}\right))-\sin (\left(\frac{11 \pi }{6}\right))-5$. Note, you could stop here and have a perfectly acceptable answer. However, you could also recall facts about the unit circle and continue on. 
\end{hint}
\begin{hint}
$f\left(\frac{11 \pi }{6}\right)=-\frac{15}{4}$.
\end{hint}
The value of the function $f(x) = \sin ^2(2 x)-\sin (x)-5$, evaluated at $x=\frac{11 \pi }{6}$, is $\answer{-\frac{15}{4}}$.
\end{solution}
\end{exercise}

\begin{exercise}
Given that $f(x)=3 \sin ^2(x)-\cos ^4(2 x)-3 \cos (2 x)+3$, evaluate $f\left(\frac{3 \pi }{2}\right)$. Express your answer in an exact form.
\begin{solution}
\begin{hint}
$f\left(\frac{3 \pi }{2}\right)=3 \sin ^2(\left(\frac{3 \pi }{2}\right))-\cos ^4(2 \left(\frac{3 \pi }{2}\right))-3 \cos (2 \left(\frac{3 \pi }{2}\right))+3$. Note, you could stop here and have a perfectly acceptable answer. However, you could also recall facts about the unit circle and continue on. 
\end{hint}
\begin{hint}
$f\left(\frac{3 \pi }{2}\right)=8$.
\end{hint}
The value of the function $f(x) = 3 \sin ^2(x)-\cos ^4(2 x)-3 \cos (2 x)+3$, evaluated at $x=\frac{3 \pi }{2}$, is $\answer{8}$.
\end{solution}
\end{exercise}

\begin{exercise}
Given that $f(x)=-\sin ^4(x)-2 \cos ^2(2 x)-5$, evaluate $f\left(\frac{11 \pi }{6}\right)$. Express your answer in an exact form.
\begin{solution}
\begin{hint}
$f\left(\frac{11 \pi }{6}\right)=-\sin ^4(\left(\frac{11 \pi }{6}\right))-2 \cos ^2(2 \left(\frac{11 \pi }{6}\right))-5$. Note, you could stop here and have a perfectly acceptable answer. However, you could also recall facts about the unit circle and continue on. 
\end{hint}
\begin{hint}
$f\left(\frac{11 \pi }{6}\right)=-\frac{89}{16}$.
\end{hint}
The value of the function $f(x) = -\sin ^4(x)-2 \cos ^2(2 x)-5$, evaluated at $x=\frac{11 \pi }{6}$, is $\answer{-\frac{89}{16}}$.
\end{solution}
\end{exercise}

\begin{exercise}
Given that $f(x)=2-4 \cos (x)$, evaluate $f\left(2 \pi\right)$. Express your answer in an exact form.
\begin{solution}
\begin{hint}
$f\left(2 \pi\right)=2-4 \cos (\left(2 \pi\right))$. Note, you could stop here and have a perfectly acceptable answer. However, you could also recall facts about the unit circle and continue on. 
\end{hint}
\begin{hint}
$f\left(2 \pi\right)=-2$.
\end{hint}
The value of the function $f(x) = 2-4 \cos (x)$, evaluated at $x=2 \pi$, is $\answer{-2}$.
\end{solution}
\end{exercise}

\begin{exercise}
Given that $f(x)=2 \cos ^2(2 x)-\cos (2 x)-5$, evaluate $f\left(\frac{5 \pi }{6}\right)$. Express your answer in an exact form.
\begin{solution}
\begin{hint}
$f\left(\frac{5 \pi }{6}\right)=2 \cos ^2(2 \left(\frac{5 \pi }{6}\right))-\cos (2 \left(\frac{5 \pi }{6}\right))-5$. Note, you could stop here and have a perfectly acceptable answer. However, you could also recall facts about the unit circle and continue on. 
\end{hint}
\begin{hint}
$f\left(\frac{5 \pi }{6}\right)=-5$.
\end{hint}
The value of the function $f(x) = 2 \cos ^2(2 x)-\cos (2 x)-5$, evaluated at $x=\frac{5 \pi }{6}$, is $\answer{-5}$.
\end{solution}
\end{exercise}

\begin{exercise}
Given that $f(x)=2 \cos ^2(2 x)-2$, evaluate $f\left(\frac{7 \pi }{4}\right)$. Express your answer in an exact form.
\begin{solution}
\begin{hint}
$f\left(\frac{7 \pi }{4}\right)=2 \cos ^2(2 \left(\frac{7 \pi }{4}\right))-2$. Note, you could stop here and have a perfectly acceptable answer. However, you could also recall facts about the unit circle and continue on. 
\end{hint}
\begin{hint}
$f\left(\frac{7 \pi }{4}\right)=-2$.
\end{hint}
The value of the function $f(x) = 2 \cos ^2(2 x)-2$, evaluated at $x=\frac{7 \pi }{4}$, is $\answer{-2}$.
\end{solution}
\end{exercise}

\begin{exercise}
Given that $f(x)=3 \cos ^2(x)+3 \cos (2 x)$, evaluate $f\left(2 \pi\right)$. Express your answer in an exact form.
\begin{solution}
\begin{hint}
$f\left(2 \pi\right)=3 \cos ^2(\left(2 \pi\right))+3 \cos (2 \left(2 \pi\right))$. Note, you could stop here and have a perfectly acceptable answer. However, you could also recall facts about the unit circle and continue on. 
\end{hint}
\begin{hint}
$f\left(2 \pi\right)=6$.
\end{hint}
The value of the function $f(x) = 3 \cos ^2(x)+3 \cos (2 x)$, evaluated at $x=2 \pi$, is $\answer{6}$.
\end{solution}
\end{exercise}

\begin{exercise}
Given that $f(x)=\sin ^4(2 x)-3 \sin ^2(2 x)+5$, evaluate $f\left(\frac{7 \pi }{6}\right)$. Express your answer in an exact form.
\begin{solution}
\begin{hint}
$f\left(\frac{7 \pi }{6}\right)=\sin ^4(2 \left(\frac{7 \pi }{6}\right))-3 \sin ^2(2 \left(\frac{7 \pi }{6}\right))+5$. Note, you could stop here and have a perfectly acceptable answer. However, you could also recall facts about the unit circle and continue on. 
\end{hint}
\begin{hint}
$f\left(\frac{7 \pi }{6}\right)=\frac{53}{16}$.
\end{hint}
The value of the function $f(x) = \sin ^4(2 x)-3 \sin ^2(2 x)+5$, evaluated at $x=\frac{7 \pi }{6}$, is $\answer{\frac{53}{16}}$.
\end{solution}
\end{exercise}

\begin{exercise}
Given that $f(x)=2 \sin ^2(2 x)+3$, evaluate $f\left(\frac{7 \pi }{4}\right)$. Express your answer in an exact form.
\begin{solution}
\begin{hint}
$f\left(\frac{7 \pi }{4}\right)=2 \sin ^2(2 \left(\frac{7 \pi }{4}\right))+3$. Note, you could stop here and have a perfectly acceptable answer. However, you could also recall facts about the unit circle and continue on. 
\end{hint}
\begin{hint}
$f\left(\frac{7 \pi }{4}\right)=5$.
\end{hint}
The value of the function $f(x) = 2 \sin ^2(2 x)+3$, evaluated at $x=\frac{7 \pi }{4}$, is $\answer{5}$.
\end{solution}
\end{exercise}

\begin{exercise}
Given that $f(x)=3 \sin ^2(2 x)$, evaluate $f\left(\frac{\pi }{4}\right)$. Express your answer in an exact form.
\begin{solution}
\begin{hint}
$f\left(\frac{\pi }{4}\right)=3 \sin ^2(2 \left(\frac{\pi }{4}\right))$. Note, you could stop here and have a perfectly acceptable answer. However, you could also recall facts about the unit circle and continue on. 
\end{hint}
\begin{hint}
$f\left(\frac{\pi }{4}\right)=3$.
\end{hint}
The value of the function $f(x) = 3 \sin ^2(2 x)$, evaluated at $x=\frac{\pi }{4}$, is $\answer{3}$.
\end{solution}
\end{exercise}

\begin{exercise}
Given that $f(x)=-2 \sin (x)+\cos ^2(2 x)-4$, evaluate $f\left(\frac{3 \pi }{4}\right)$. Express your answer in an exact form.
\begin{solution}
\begin{hint}
$f\left(\frac{3 \pi }{4}\right)=-2 \sin (\left(\frac{3 \pi }{4}\right))+\cos ^2(2 \left(\frac{3 \pi }{4}\right))-4$. Note, you could stop here and have a perfectly acceptable answer. However, you could also recall facts about the unit circle and continue on. 
\end{hint}
\begin{hint}
$f\left(\frac{3 \pi }{4}\right)=-4-\sqrt{2}$.
\end{hint}
The value of the function $f(x) = -2 \sin (x)+\cos ^2(2 x)-4$, evaluated at $x=\frac{3 \pi }{4}$, is $\answer{-4-\sqrt{2}}$.
\end{solution}
\end{exercise}

\begin{exercise}
Given that $f(x)=\sin ^3(x)+\sin (2 x)+2$, evaluate $f\left(\frac{7 \pi }{4}\right)$. Express your answer in an exact form.
\begin{solution}
\begin{hint}
$f\left(\frac{7 \pi }{4}\right)=\sin ^3(\left(\frac{7 \pi }{4}\right))+\sin (2 \left(\frac{7 \pi }{4}\right))+2$. Note, you could stop here and have a perfectly acceptable answer. However, you could also recall facts about the unit circle and continue on. 
\end{hint}
\begin{hint}
$f\left(\frac{7 \pi }{4}\right)=1-\frac{1}{2 \sqrt{2}}$.
\end{hint}
The value of the function $f(x) = \sin ^3(x)+\sin (2 x)+2$, evaluated at $x=\frac{7 \pi }{4}$, is $\answer{1-\frac{1}{2 \sqrt{2}}}$.
\end{solution}
\end{exercise}

\begin{exercise}
Given that $f(x)=2 \cos ^3(x)+3 \cos ^2(2 x)+2$, evaluate $f\left(\frac{11 \pi }{6}\right)$. Express your answer in an exact form.
\begin{solution}
\begin{hint}
$f\left(\frac{11 \pi }{6}\right)=2 \cos ^3(\left(\frac{11 \pi }{6}\right))+3 \cos ^2(2 \left(\frac{11 \pi }{6}\right))+2$. Note, you could stop here and have a perfectly acceptable answer. However, you could also recall facts about the unit circle and continue on. 
\end{hint}
\begin{hint}
$f\left(\frac{11 \pi }{6}\right)=\frac{11}{4}+\frac{3 \sqrt{3}}{4}$.
\end{hint}
The value of the function $f(x) = 2 \cos ^3(x)+3 \cos ^2(2 x)+2$, evaluated at $x=\frac{11 \pi }{6}$, is $\answer{\frac{11}{4}+\frac{3 \sqrt{3}}{4}}$.
\end{solution}
\end{exercise}

\begin{exercise}
Given that $f(x)=4$, evaluate $f\left(\frac{5 \pi }{4}\right)$. Express your answer in an exact form.
\begin{solution}
\begin{hint}
$f\left(\frac{5 \pi }{4}\right)=4$. Note, you could stop here and have a perfectly acceptable answer. However, you could also recall facts about the unit circle and continue on. 
\end{hint}
\begin{hint}
$f\left(\frac{5 \pi }{4}\right)=4$.
\end{hint}
The value of the function $f(x) = 4$, evaluated at $x=\frac{5 \pi }{4}$, is $\answer{4}$.
\end{solution}
\end{exercise}

\begin{exercise}
Given that $f(x)=4-\cos (x)$, evaluate $f\left(\frac{4 \pi }{3}\right)$. Express your answer in an exact form.
\begin{solution}
\begin{hint}
$f\left(\frac{4 \pi }{3}\right)=4-\cos (\left(\frac{4 \pi }{3}\right))$. Note, you could stop here and have a perfectly acceptable answer. However, you could also recall facts about the unit circle and continue on. 
\end{hint}
\begin{hint}
$f\left(\frac{4 \pi }{3}\right)=\frac{9}{2}$.
\end{hint}
The value of the function $f(x) = 4-\cos (x)$, evaluated at $x=\frac{4 \pi }{3}$, is $\answer{\frac{9}{2}}$.
\end{solution}
\end{exercise}
\end{shuffle}

%%%%%%%%%%%%%%%%%%%%%%%%%%%%%%%%%%%%%%%%%%%%%%%%%%%%%%%%%%%%
%%%%%%%%%%%%%%%%%%%%%%%%%%%%%%%%%%%%%%%%%%%%%%%%%%%%%%%%%%%%
%% Problem
%%%%%%%%%%%%%%%%%%%%%%%%%%%%%%%%%%%%%%%%%%%%%%%%%%%%%%%%%%%%
%%%%%%%%%%%%%%%%%%%%%%%%%%%%%%%%%%%%%%%%%%%%%%%%%%%%%%%%%%%%

\begin{shuffle}
\begin{exercise}
Given that $f(x)=\sqrt{4 x^2-4 x+5}$, evaluate $f\left(\frac{24}{5}\right)$. Express your answer in an exact form.
\begin{solution}
\begin{hint}
$f\left(\frac{24}{5}\right)=\sqrt{4 (\frac{24}{5})^2-4 (\frac{24}{5})+5}$. Note, you could stop here and have a perfectly acceptable answer. However, we can simplify a bit more. 
\end{hint}
\begin{hint}
$f\left(\frac{24}{5}\right)=\frac{\sqrt{1949}}{5}$.
\end{hint}
The value of the function $f(x) = \sqrt{4 x^2-4 x+5}$, evaluated at $x=\frac{24}{5}$, is $\answer{\frac{\sqrt{1949}}{5}}$.
\end{solution}
\end{exercise}

\begin{exercise}
Given that $f(x)=\sqrt{5 x^2+x+2}$, evaluate $f\left(-\frac{14}{5}\right)$. Express your answer in an exact form.
\begin{solution}
\begin{hint}
$f\left(-\frac{14}{5}\right)=\sqrt{5 (-\frac{14}{5})^2+(-\frac{14}{5})+2}$. Note, you could stop here and have a perfectly acceptable answer. However, we can simplify a bit more. 
\end{hint}
\begin{hint}
$f\left(-\frac{14}{5}\right)=8 \sqrt{\frac{3}{5}}$.
\end{hint}
The value of the function $f(x) = \sqrt{5 x^2+x+2}$, evaluated at $x=-\frac{14}{5}$, is $\answer{8 \sqrt{\frac{3}{5}}}$.
\end{solution}
\end{exercise}

\begin{exercise}
Given that $f(x)=\sqrt{3 x^2+2 x+3}$, evaluate $f\left(-\frac{3}{5}\right)$. Express your answer in an exact form.
\begin{solution}
\begin{hint}
$f\left(-\frac{3}{5}\right)=\sqrt{3 (-\frac{3}{5})^2+2 (-\frac{3}{5})+3}$. Note, you could stop here and have a perfectly acceptable answer. However, we can simplify a bit more. 
\end{hint}
\begin{hint}
$f\left(-\frac{3}{5}\right)=\frac{6 \sqrt{2}}{5}$.
\end{hint}
The value of the function $f(x) = \sqrt{3 x^2+2 x+3}$, evaluated at $x=-\frac{3}{5}$, is $\answer{\frac{6 \sqrt{2}}{5}}$.
\end{solution}
\end{exercise}

\begin{exercise}
Given that $f(x)=\sqrt{x^2+2 x-4}$, evaluate $f\left(\frac{9}{5}\right)$. Express your answer in an exact form.
\begin{solution}
\begin{hint}
$f\left(\frac{9}{5}\right)=\sqrt{(\frac{9}{5})^2+2 (\frac{9}{5})-4}$. Note, you could stop here and have a perfectly acceptable answer. However, we can simplify a bit more. 
\end{hint}
\begin{hint}
$f\left(\frac{9}{5}\right)=\frac{\sqrt{71}}{5}$.
\end{hint}
The value of the function $f(x) = \sqrt{x^2+2 x-4}$, evaluated at $x=\frac{9}{5}$, is $\answer{\frac{\sqrt{71}}{5}}$.
\end{solution}
\end{exercise}

\begin{exercise}
Given that $f(x)=\sqrt{3 x^2+3 x-4}$, evaluate $f\left(\frac{7}{5}\right)$. Express your answer in an exact form.
\begin{solution}
\begin{hint}
$f\left(\frac{7}{5}\right)=\sqrt{3 (\frac{7}{5})^2+3 (\frac{7}{5})-4}$. Note, you could stop here and have a perfectly acceptable answer. However, we can simplify a bit more. 
\end{hint}
\begin{hint}
$f\left(\frac{7}{5}\right)=\frac{2 \sqrt{38}}{5}$.
\end{hint}
The value of the function $f(x) = \sqrt{3 x^2+3 x-4}$, evaluated at $x=\frac{7}{5}$, is $\answer{\frac{2 \sqrt{38}}{5}}$.
\end{solution}
\end{exercise}

\begin{exercise}
Given that $f(x)=\sqrt{-3 x^2-4 x+4}$, evaluate $f\left(-\frac{2}{5}\right)$. Express your answer in an exact form.
\begin{solution}
\begin{hint}
$f\left(-\frac{2}{5}\right)=\sqrt{-3 (-\frac{2}{5})^2-4 (-\frac{2}{5})+4}$. Note, you could stop here and have a perfectly acceptable answer. However, we can simplify a bit more. 
\end{hint}
\begin{hint}
$f\left(-\frac{2}{5}\right)=\frac{8 \sqrt{2}}{5}$.
\end{hint}
The value of the function $f(x) = \sqrt{-3 x^2-4 x+4}$, evaluated at $x=-\frac{2}{5}$, is $\answer{\frac{8 \sqrt{2}}{5}}$.
\end{solution}
\end{exercise}

\begin{exercise}
Given that $f(x)=\sqrt{2 x^2-2 x-1}$, evaluate $f\left(-\frac{47}{10}\right)$. Express your answer in an exact form.
\begin{solution}
\begin{hint}
$f\left(-\frac{47}{10}\right)=\sqrt{2 (-\frac{47}{10})^2-2 (-\frac{47}{10})-1}$. Note, you could stop here and have a perfectly acceptable answer. However, we can simplify a bit more. 
\end{hint}
\begin{hint}
$f\left(-\frac{47}{10}\right)=\frac{\sqrt{\frac{2629}{2}}}{5}$.
\end{hint}
The value of the function $f(x) = \sqrt{2 x^2-2 x-1}$, evaluated at $x=-\frac{47}{10}$, is $\answer{\frac{\sqrt{\frac{2629}{2}}}{5}}$.
\end{solution}
\end{exercise}

\begin{exercise}
Given that $f(x)=\sqrt{3 x^2+5 x-1}$, evaluate $f\left(-\frac{24}{5}\right)$. Express your answer in an exact form.
\begin{solution}
\begin{hint}
$f\left(-\frac{24}{5}\right)=\sqrt{3 (-\frac{24}{5})^2+5 (-\frac{24}{5})-1}$. Note, you could stop here and have a perfectly acceptable answer. However, we can simplify a bit more. 
\end{hint}
\begin{hint}
$f\left(-\frac{24}{5}\right)=\frac{\sqrt{1103}}{5}$.
\end{hint}
The value of the function $f(x) = \sqrt{3 x^2+5 x-1}$, evaluated at $x=-\frac{24}{5}$, is $\answer{\frac{\sqrt{1103}}{5}}$.
\end{solution}
\end{exercise}

\begin{exercise}
Given that $f(x)=\sqrt{4 x^2+4 x-5}$, evaluate $f\left(\frac{9}{2}\right)$. Express your answer in an exact form.
\begin{solution}
\begin{hint}
$f\left(\frac{9}{2}\right)=\sqrt{4 (\frac{9}{2})^2+4 (\frac{9}{2})-5}$. Note, you could stop here and have a perfectly acceptable answer. However, we can simplify a bit more. 
\end{hint}
\begin{hint}
$f\left(\frac{9}{2}\right)=\sqrt{94}$.
\end{hint}
The value of the function $f(x) = \sqrt{4 x^2+4 x-5}$, evaluated at $x=\frac{9}{2}$, is $\answer{\sqrt{94}}$.
\end{solution}
\end{exercise}

\begin{exercise}
Given that $f(x)=\sqrt{x^2+x}$, evaluate $f\left(\frac{19}{5}\right)$. Express your answer in an exact form.
\begin{solution}
\begin{hint}
$f\left(\frac{19}{5}\right)=\sqrt{(\frac{19}{5})^2+(\frac{19}{5})}$. Note, you could stop here and have a perfectly acceptable answer. However, we can simplify a bit more. 
\end{hint}
\begin{hint}
$f\left(\frac{19}{5}\right)=\frac{2 \sqrt{114}}{5}$.
\end{hint}
The value of the function $f(x) = \sqrt{x^2+x}$, evaluated at $x=\frac{19}{5}$, is $\answer{\frac{2 \sqrt{114}}{5}}$.
\end{solution}
\end{exercise}

\begin{exercise}
Given that $f(x)=\sqrt{3 x^2+3 x-2}$, evaluate $f\left(\frac{5}{2}\right)$. Express your answer in an exact form.
\begin{solution}
\begin{hint}
$f\left(\frac{5}{2}\right)=\sqrt{3 (\frac{5}{2})^2+3 (\frac{5}{2})-2}$. Note, you could stop here and have a perfectly acceptable answer. However, we can simplify a bit more. 
\end{hint}
\begin{hint}
$f\left(\frac{5}{2}\right)=\frac{\sqrt{97}}{2}$.
\end{hint}
The value of the function $f(x) = \sqrt{3 x^2+3 x-2}$, evaluated at $x=\frac{5}{2}$, is $\answer{\frac{\sqrt{97}}{2}}$.
\end{solution}
\end{exercise}

\begin{exercise}
Given that $f(x)=\sqrt{3 x^2+x-2}$, evaluate $f\left(\frac{27}{10}\right)$. Express your answer in an exact form.
\begin{solution}
\begin{hint}
$f\left(\frac{27}{10}\right)=\sqrt{3 (\frac{27}{10})^2+(\frac{27}{10})-2}$. Note, you could stop here and have a perfectly acceptable answer. However, we can simplify a bit more. 
\end{hint}
\begin{hint}
$f\left(\frac{27}{10}\right)=\frac{\sqrt{2257}}{10}$.
\end{hint}
The value of the function $f(x) = \sqrt{3 x^2+x-2}$, evaluated at $x=\frac{27}{10}$, is $\answer{\frac{\sqrt{2257}}{10}}$.
\end{solution}
\end{exercise}

\begin{exercise}
Given that $f(x)=\sqrt{3 x^2+x}$, evaluate $f\left(\frac{9}{10}\right)$. Express your answer in an exact form.
\begin{solution}
\begin{hint}
$f\left(\frac{9}{10}\right)=\sqrt{3 (\frac{9}{10})^2+(\frac{9}{10})}$. Note, you could stop here and have a perfectly acceptable answer. However, we can simplify a bit more. 
\end{hint}
\begin{hint}
$f\left(\frac{9}{10}\right)=\frac{3 \sqrt{37}}{10}$.
\end{hint}
The value of the function $f(x) = \sqrt{3 x^2+x}$, evaluated at $x=\frac{9}{10}$, is $\answer{\frac{3 \sqrt{37}}{10}}$.
\end{solution}
\end{exercise}

\begin{exercise}
Given that $f(x)=\sqrt{-4 x-5}$, evaluate $f\left(-\frac{23}{10}\right)$. Express your answer in an exact form.
\begin{solution}
\begin{hint}
$f\left(-\frac{23}{10}\right)=\sqrt{-4 (-\frac{23}{10})-5}$. Note, you could stop here and have a perfectly acceptable answer. However, we can simplify a bit more. 
\end{hint}
\begin{hint}
$f\left(-\frac{23}{10}\right)=\sqrt{\frac{21}{5}}$.
\end{hint}
The value of the function $f(x) = \sqrt{-4 x-5}$, evaluated at $x=-\frac{23}{10}$, is $\answer{\sqrt{\frac{21}{5}}}$.
\end{solution}
\end{exercise}

\begin{exercise}
Given that $f(x)=\sqrt{x^2-5 x+4}$, evaluate $f\left(-\frac{19}{10}\right)$. Express your answer in an exact form.
\begin{solution}
\begin{hint}
$f\left(-\frac{19}{10}\right)=\sqrt{(-\frac{19}{10})^2-5 (-\frac{19}{10})+4}$. Note, you could stop here and have a perfectly acceptable answer. However, we can simplify a bit more. 
\end{hint}
\begin{hint}
$f\left(-\frac{19}{10}\right)=\frac{\sqrt{1711}}{10}$.
\end{hint}
The value of the function $f(x) = \sqrt{x^2-5 x+4}$, evaluated at $x=-\frac{19}{10}$, is $\answer{\frac{\sqrt{1711}}{10}}$.
\end{solution}
\end{exercise}

\begin{exercise}
Given that $f(x)=\sqrt{-x^2-2 x+2}$, evaluate $f\left(-\frac{9}{5}\right)$. Express your answer in an exact form.
\begin{solution}
\begin{hint}
$f\left(-\frac{9}{5}\right)=\sqrt{-(-\frac{9}{5})^2-2 (-\frac{9}{5})+2}$. Note, you could stop here and have a perfectly acceptable answer. However, we can simplify a bit more. 
\end{hint}
\begin{hint}
$f\left(-\frac{9}{5}\right)=\frac{\sqrt{59}}{5}$.
\end{hint}
The value of the function $f(x) = \sqrt{-x^2-2 x+2}$, evaluated at $x=-\frac{9}{5}$, is $\answer{\frac{\sqrt{59}}{5}}$.
\end{solution}
\end{exercise}

\begin{exercise}
Given that $f(x)=\sqrt{2 x^2+5 x+2}$, evaluate $f\left(-\frac{18}{5}\right)$. Express your answer in an exact form.
\begin{solution}
\begin{hint}
$f\left(-\frac{18}{5}\right)=\sqrt{2 (-\frac{18}{5})^2+5 (-\frac{18}{5})+2}$. Note, you could stop here and have a perfectly acceptable answer. However, we can simplify a bit more. 
\end{hint}
\begin{hint}
$f\left(-\frac{18}{5}\right)=\frac{2 \sqrt{62}}{5}$.
\end{hint}
The value of the function $f(x) = \sqrt{2 x^2+5 x+2}$, evaluated at $x=-\frac{18}{5}$, is $\answer{\frac{2 \sqrt{62}}{5}}$.
\end{solution}
\end{exercise}

\begin{exercise}
Given that $f(x)=\sqrt{5 x^2+4 x+5}$, evaluate $f\left(-\frac{1}{2}\right)$. Express your answer in an exact form.
\begin{solution}
\begin{hint}
$f\left(-\frac{1}{2}\right)=\sqrt{5 (-\frac{1}{2})^2+4 (-\frac{1}{2})+5}$. Note, you could stop here and have a perfectly acceptable answer. However, we can simplify a bit more. 
\end{hint}
\begin{hint}
$f\left(-\frac{1}{2}\right)=\frac{\sqrt{17}}{2}$.
\end{hint}
The value of the function $f(x) = \sqrt{5 x^2+4 x+5}$, evaluated at $x=-\frac{1}{2}$, is $\answer{\frac{\sqrt{17}}{2}}$.
\end{solution}
\end{exercise}

\begin{exercise}
Given that $f(x)=\sqrt{-x^2-4 x+2}$, evaluate $f\left(-\frac{18}{5}\right)$. Express your answer in an exact form.
\begin{solution}
\begin{hint}
$f\left(-\frac{18}{5}\right)=\sqrt{-(-\frac{18}{5})^2-4 (-\frac{18}{5})+2}$. Note, you could stop here and have a perfectly acceptable answer. However, we can simplify a bit more. 
\end{hint}
\begin{hint}
$f\left(-\frac{18}{5}\right)=\frac{\sqrt{86}}{5}$.
\end{hint}
The value of the function $f(x) = \sqrt{-x^2-4 x+2}$, evaluated at $x=-\frac{18}{5}$, is $\answer{\frac{\sqrt{86}}{5}}$.
\end{solution}
\end{exercise}

\begin{exercise}
Given that $f(x)=\sqrt{3 x^2+x-5}$, evaluate $f\left(-5\right)$. Express your answer in an exact form.
\begin{solution}
\begin{hint}
$f\left(-5\right)=\sqrt{3 (-5)^2+(-5)-5}$. Note, you could stop here and have a perfectly acceptable answer. However, we can simplify a bit more. 
\end{hint}
\begin{hint}
$f\left(-5\right)=\sqrt{65}$.
\end{hint}
The value of the function $f(x) = \sqrt{3 x^2+x-5}$, evaluated at $x=-5$, is $\answer{\sqrt{65}}$.
\end{solution}
\end{exercise}
\end{shuffle}


%%%%%%%%%%%%%%%%%%%%%%%%%%%%%%%%%%%%%%%%%%%%%%%%%%%%%%%%%%%%
%%%%%%%%%%%%%%%%%%%%%%%%%%%%%%%%%%%%%%%%%%%%%%%%%%%%%%%%%%%%
%% Problem
%%%%%%%%%%%%%%%%%%%%%%%%%%%%%%%%%%%%%%%%%%%%%%%%%%%%%%%%%%%%
%%%%%%%%%%%%%%%%%%%%%%%%%%%%%%%%%%%%%%%%%%%%%%%%%%%%%%%%%%%%


\begin{shuffle}
\begin{exercise}
Given that $f(x)=-x^2+3 x-1$, evaluate $f(x+h)-f(x)$.
\begin{solution}
\begin{hint}
$f(x+h)-f(x)=(-(h+x)^2+3 (h+x)-1)-(-x^2+3 x-1)$. Note, you could stop here and have a perfectly acceptable answer. However, we can expand this out.
\end{hint}
\begin{hint}
$f(x+h)-f(x)=-(h+x)^2+3 (h+x)+x^2-3 x$.
\end{hint}
The value of the function $f(x+h)-f(x)$, is $\answer{-h^2-2 h x+3 h}$.
\end{solution}
\end{exercise}

\begin{exercise}
Given that $f(x)=-x^4+2 x^2-x+5$, evaluate $f(x+h)-f(x)$.
\begin{solution}
\begin{hint}
$f(x+h)-f(x)=(-(h+x)^4+2 (h+x)^2-h-x+5)-(-x^4+2 x^2-x+5)$. Note, you could stop here and have a perfectly acceptable answer. However, we can expand this out.
\end{hint}
\begin{hint}
$f(x+h)-f(x)=-(h+x)^4+2 (h+x)^2-h+x^4-2 x^2$.
\end{hint}
The value of the function $f(x+h)-f(x)$, is $\answer{-h^4-4 h^3 x-6 h^2 x^2+2 h^2-4 h x^3+4 h x-h}$.
\end{solution}
\end{exercise}

\begin{exercise}
Given that $f(x)=x^2-2 x+2$, evaluate $f(x+h)-f(x)$.
\begin{solution}
\begin{hint}
$f(x+h)-f(x)=((h+x)^2-2 (h+x)+2)-(x^2-2 x+2)$. Note, you could stop here and have a perfectly acceptable answer. However, we can expand this out.
\end{hint}
\begin{hint}
$f(x+h)-f(x)=(h+x)^2-2 (h+x)-x^2+2 x$.
\end{hint}
The value of the function $f(x+h)-f(x)$, is $\answer{h^2+2 h x-2 h}$.
\end{solution}
\end{exercise}

\begin{exercise}
Given that $f(x)=x^4+2 x^2-x+2$, evaluate $f(x+h)-f(x)$.
\begin{solution}
\begin{hint}
$f(x+h)-f(x)=((h+x)^4+2 (h+x)^2-h-x+2)-(x^4+2 x^2-x+2)$. Note, you could stop here and have a perfectly acceptable answer. However, we can expand this out.
\end{hint}
\begin{hint}
$f(x+h)-f(x)=(h+x)^4+2 (h+x)^2-h-x^4-2 x^2$.
\end{hint}
The value of the function $f(x+h)-f(x)$, is $\answer{h^4+4 h^3 x+6 h^2 x^2+2 h^2+4 h x^3+4 h x-h}$.
\end{solution}
\end{exercise}

\begin{exercise}
Given that $f(x)=-3 x^2+2 x-4$, evaluate $f(x+h)-f(x)$.
\begin{solution}
\begin{hint}
$f(x+h)-f(x)=(-3 (h+x)^2+2 (h+x)-4)-(-3 x^2+2 x-4)$. Note, you could stop here and have a perfectly acceptable answer. However, we can expand this out.
\end{hint}
\begin{hint}
$f(x+h)-f(x)=-3 (h+x)^2+2 (h+x)+3 x^2-2 x$.
\end{hint}
The value of the function $f(x+h)-f(x)$, is $\answer{-3 h^2-6 h x+2 h}$.
\end{solution}
\end{exercise}

\begin{exercise}
Given that $f(x)=x^2+2 x$, evaluate $f(x+h)-f(x)$.
\begin{solution}
\begin{hint}
$f(x+h)-f(x)=((h+x)^2+2 (h+x))-(x^2+2 x)$. Note, you could stop here and have a perfectly acceptable answer. However, we can expand this out.
\end{hint}
\begin{hint}
$f(x+h)-f(x)=(h+x)^2+2 (h+x)-x^2-2 x$.
\end{hint}
The value of the function $f(x+h)-f(x)$, is $\answer{h^2+2 h x+2 h}$.
\end{solution}
\end{exercise}

\begin{exercise}
Given that $f(x)=x^2-2 x$, evaluate $f(x+h)-f(x)$.
\begin{solution}
\begin{hint}
$f(x+h)-f(x)=((h+x)^2-2 (h+x))-(x^2-2 x)$. Note, you could stop here and have a perfectly acceptable answer. However, we can expand this out.
\end{hint}
\begin{hint}
$f(x+h)-f(x)=(h+x)^2-2 (h+x)-x^2+2 x$.
\end{hint}
The value of the function $f(x+h)-f(x)$, is $\answer{h^2+2 h x-2 h}$.
\end{solution}
\end{exercise}

\begin{exercise}
Given that $f(x)=3 x^2-2 x+3$, evaluate $f(x+h)-f(x)$.
\begin{solution}
\begin{hint}
$f(x+h)-f(x)=(3 (h+x)^2-2 (h+x)+3)-(3 x^2-2 x+3)$. Note, you could stop here and have a perfectly acceptable answer. However, we can expand this out.
\end{hint}
\begin{hint}
$f(x+h)-f(x)=3 (h+x)^2-2 (h+x)-3 x^2+2 x$.
\end{hint}
The value of the function $f(x+h)-f(x)$, is $\answer{3 h^2+6 h x-2 h}$.
\end{solution}
\end{exercise}

\begin{exercise}
Given that $f(x)=2 x^3-3 x^2-3 x+1$, evaluate $f(x+h)-f(x)$.
\begin{solution}
\begin{hint}
$f(x+h)-f(x)=(2 (h+x)^3-3 (h+x)^2-3 (h+x)+1)-(2 x^3-3 x^2-3 x+1)$. Note, you could stop here and have a perfectly acceptable answer. However, we can expand this out.
\end{hint}
\begin{hint}
$f(x+h)-f(x)=2 (h+x)^3-3 (h+x)^2-3 (h+x)-2 x^3+3 x^2+3 x$.
\end{hint}
The value of the function $f(x+h)-f(x)$, is $\answer{2 h^3+6 h^2 x-3 h^2+6 h x^2-6 h x-3 h}$.
\end{solution}
\end{exercise}

\begin{exercise}
Given that $f(x)=-3 x-3$, evaluate $f(x+h)-f(x)$.
\begin{solution}
\begin{hint}
$f(x+h)-f(x)=(-3 (h+x)-3)-(-3 x-3)$. Note, you could stop here and have a perfectly acceptable answer. However, we can expand this out.
\end{hint}
\begin{hint}
$f(x+h)-f(x)=3 x-3 (h+x)$.
\end{hint}
The value of the function $f(x+h)-f(x)$, is $\answer{-3 h}$.
\end{solution}
\end{exercise}

\begin{exercise}
Given that $f(x)=x^3+x^2-x-4$, evaluate $f(x+h)-f(x)$.
\begin{solution}
\begin{hint}
$f(x+h)-f(x)=((h+x)^3+(h+x)^2-h-x-4)-(x^3+x^2-x-4)$. Note, you could stop here and have a perfectly acceptable answer. However, we can expand this out.
\end{hint}
\begin{hint}
$f(x+h)-f(x)=(h+x)^3+(h+x)^2-h-x^3-x^2$.
\end{hint}
The value of the function $f(x+h)-f(x)$, is $\answer{h^3+3 h^2 x+h^2+3 h x^2+2 h x-h}$.
\end{solution}
\end{exercise}

\begin{exercise}
Given that $f(x)=5-4 x$, evaluate $f(x+h)-f(x)$.
\begin{solution}
\begin{hint}
$f(x+h)-f(x)=(5-4 (h+x))-(5-4 x)$. Note, you could stop here and have a perfectly acceptable answer. However, we can expand this out.
\end{hint}
\begin{hint}
$f(x+h)-f(x)=4 x-4 (h+x)$.
\end{hint}
The value of the function $f(x+h)-f(x)$, is $\answer{-4 h}$.
\end{solution}
\end{exercise}

\begin{exercise}
Given that $f(x)=-x^3-2 x^2-4$, evaluate $f(x+h)-f(x)$.
\begin{solution}
\begin{hint}
$f(x+h)-f(x)=(-(h+x)^3-2 (h+x)^2-4)-(-x^3-2 x^2-4)$. Note, you could stop here and have a perfectly acceptable answer. However, we can expand this out.
\end{hint}
\begin{hint}
$f(x+h)-f(x)=-(h+x)^3-2 (h+x)^2+x^3+2 x^2$.
\end{hint}
The value of the function $f(x+h)-f(x)$, is $\answer{-h^3-3 h^2 x-2 h^2-3 h x^2-4 h x}$.
\end{solution}
\end{exercise}

\begin{exercise}
Given that $f(x)=-x^4-x^2-3 x+2$, evaluate $f(x+h)-f(x)$.
\begin{solution}
\begin{hint}
$f(x+h)-f(x)=(-(h+x)^4-(h+x)^2-3 (h+x)+2)-(-x^4-x^2-3 x+2)$. Note, you could stop here and have a perfectly acceptable answer. However, we can expand this out.
\end{hint}
\begin{hint}
$f(x+h)-f(x)=-(h+x)^4-(h+x)^2-3 (h+x)+x^4+x^2+3 x$.
\end{hint}
The value of the function $f(x+h)-f(x)$, is $\answer{-h^4-4 h^3 x-6 h^2 x^2-h^2-4 h x^3-2 h x-3 h}$.
\end{solution}
\end{exercise}

\begin{exercise}
Given that $f(x)=x^3+x^2-2 x-2$, evaluate $f(x+h)-f(x)$.
\begin{solution}
\begin{hint}
$f(x+h)-f(x)=((h+x)^3+(h+x)^2-2 (h+x)-2)-(x^3+x^2-2 x-2)$. Note, you could stop here and have a perfectly acceptable answer. However, we can expand this out.
\end{hint}
\begin{hint}
$f(x+h)-f(x)=(h+x)^3+(h+x)^2-2 (h+x)-x^3-x^2+2 x$.
\end{hint}
The value of the function $f(x+h)-f(x)$, is $\answer{h^3+3 h^2 x+h^2+3 h x^2+2 h x-2 h}$.
\end{solution}
\end{exercise}

\begin{exercise}
Given that $f(x)=-2 x^2+2 x+5$, evaluate $f(x+h)-f(x)$.
\begin{solution}
\begin{hint}
$f(x+h)-f(x)=(-2 (h+x)^2+2 (h+x)+5)-(-2 x^2+2 x+5)$. Note, you could stop here and have a perfectly acceptable answer. However, we can expand this out.
\end{hint}
\begin{hint}
$f(x+h)-f(x)=-2 (h+x)^2+2 (h+x)+2 x^2-2 x$.
\end{hint}
The value of the function $f(x+h)-f(x)$, is $\answer{-2 h^2-4 h x+2 h}$.
\end{solution}
\end{exercise}

\begin{exercise}
Given that $f(x)=2 x^2-3 x-1$, evaluate $f(x+h)-f(x)$.
\begin{solution}
\begin{hint}
$f(x+h)-f(x)=(2 (h+x)^2-3 (h+x)-1)-(2 x^2-3 x-1)$. Note, you could stop here and have a perfectly acceptable answer. However, we can expand this out.
\end{hint}
\begin{hint}
$f(x+h)-f(x)=2 (h+x)^2-3 (h+x)-2 x^2+3 x$.
\end{hint}
The value of the function $f(x+h)-f(x)$, is $\answer{2 h^2+4 h x-3 h}$.
\end{solution}
\end{exercise}

\begin{exercise}
Given that $f(x)=3 x^2+x+4$, evaluate $f(x+h)-f(x)$.
\begin{solution}
\begin{hint}
$f(x+h)-f(x)=(3 (h+x)^2+h+x+4)-(3 x^2+x+4)$. Note, you could stop here and have a perfectly acceptable answer. However, we can expand this out.
\end{hint}
\begin{hint}
$f(x+h)-f(x)=3 (h+x)^2+h-3 x^2$.
\end{hint}
The value of the function $f(x+h)-f(x)$, is $\answer{3 h^2+6 h x+h}$.
\end{solution}
\end{exercise}

\begin{exercise}
Given that $f(x)=2 x^2-4 x+4$, evaluate $f(x+h)-f(x)$.
\begin{solution}
\begin{hint}
$f(x+h)-f(x)=(2 (h+x)^2-4 (h+x)+4)-(2 x^2-4 x+4)$. Note, you could stop here and have a perfectly acceptable answer. However, we can expand this out.
\end{hint}
\begin{hint}
$f(x+h)-f(x)=2 (h+x)^2-4 (h+x)-2 x^2+4 x$.
\end{hint}
The value of the function $f(x+h)-f(x)$, is $\answer{2 h^2+4 h x-4 h}$.
\end{solution}
\end{exercise}

\begin{exercise}
Given that $f(x)=x^4+x^3+4$, evaluate $f(x+h)-f(x)$.
\begin{solution}
\begin{hint}
$f(x+h)-f(x)=((h+x)^4+(h+x)^3+4)-(x^4+x^3+4)$. Note, you could stop here and have a perfectly acceptable answer. However, we can expand this out.
\end{hint}
\begin{hint}
$f(x+h)-f(x)=(h+x)^4+(h+x)^3-x^4-x^3$.
\end{hint}
The value of the function $f(x+h)-f(x)$, is $\answer{h^4+4 h^3 x+h^3+6 h^2 x^2+3 h^2 x+4 h x^3+3 h x^2}$.
\end{solution}
\end{exercise}
\end{shuffle}
