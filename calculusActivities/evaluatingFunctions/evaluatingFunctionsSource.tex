\headline{In this activity we practice evaluating functions at numbers and other functions.} % A one sentence description of the activity
\activitytitle{Evaluating Functions} % the title of the activity

We'll start off easy, asking you a set of questions that you can
probably do.

\begin{shuffle}
\begin{exercise}
Given that $f(x)=x^4+2 x^2+4 x+5$, evaluate $f(0.5)$. Express your answer in decimal notation.
\begin{solution}
\begin{hint}
$f(0.5)=(0.5)^4+2 (0.5)^2+4 (0.5)+5$.
\end{hint}
\begin{hint}
$f(0.5)=7.5625$.
\end{hint}
The value of the function $f(x) = x^4+2 x^2+4 x+5$, evaluated at $x=0.5$, is $\answer{7.5625}$.
\end{solution}
\end{exercise}

\begin{exercise}
Given that $f(x)=x^4+2 x^3-2 x^2-4 x-2$, evaluate $f(-1)$. Express your answer in decimal notation.
\begin{solution}
\begin{hint}
$f(-1)=(-1)^4+2 (-1)^3-2 (-1)^2-4 (-1)-2$.
\end{hint}
\begin{hint}
$f(-1)=-1$.
\end{hint}
The value of the function $f(x) = x^4+2 x^3-2 x^2-4 x-2$, evaluated at $x=-1$, is $\answer{-1}$.
\end{solution}
\end{exercise}

\begin{exercise}
Given that $f(x)=2 x^2+2 x-5$, evaluate $f(-5)$. Express your answer in decimal notation.
\begin{solution}
\begin{hint}
$f(-5)=2 (-5)^2+2 (-5)-5$.
\end{hint}
\begin{hint}
$f(-5)=35$.
\end{hint}
The value of the function $f(x) = 2 x^2+2 x-5$, evaluated at $x=-5$, is $\answer{35}$.
\end{solution}
\end{exercise}

\begin{exercise}
Given that $f(x)=x^4+2 x^2+3 x-2$, evaluate $f(-2.8)$. Express your answer in decimal notation.
\begin{solution}
\begin{hint}
$f(-2.8)=(-2.8)^4+2 (-2.8)^2+3 (-2.8)-2$.
\end{hint}
\begin{hint}
$f(-2.8)=66.7456$.
\end{hint}
The value of the function $f(x) = x^4+2 x^2+3 x-2$, evaluated at $x=-2.8$, is $\answer{66.7456}$.
\end{solution}
\end{exercise}

\begin{exercise}
Given that $f(x)=-x^2-x+4$, evaluate $f(-3.1)$. Express your answer in decimal notation.
\begin{solution}
\begin{hint}
$f(-3.1)=-(-3.1)^2-(-3.1)+4$.
\end{hint}
\begin{hint}
$f(-3.1)=-2.51$.
\end{hint}
The value of the function $f(x) = -x^2-x+4$, evaluated at $x=-3.1$, is $\answer{-2.51}$.
\end{solution}
\end{exercise}

\begin{exercise}
Given that $f(x)=-3 x^2-4 x+4$, evaluate $f(-2)$. Express your answer in decimal notation.
\begin{solution}
\begin{hint}
$f(-2)=-3 (-2)^2-4 (-2)+4$.
\end{hint}
\begin{hint}
$f(-2)=0$.
\end{hint}
The value of the function $f(x) = -3 x^2-4 x+4$, evaluated at $x=-2$, is $\answer{0}$.
\end{solution}
\end{exercise}

\begin{exercise}
Given that $f(x)=-2 x^3+x^2+3$, evaluate $f(4.2)$. Express your answer in decimal notation.
\begin{solution}
\begin{hint}
$f(4.2)=-2 (4.2)^3+(4.2)^2+3$.
\end{hint}
\begin{hint}
$f(4.2)=-127.536$.
\end{hint}
The value of the function $f(x) = -2 x^3+x^2+3$, evaluated at $x=4.2$, is $\answer{-127.536}$.
\end{solution}
\end{exercise}

\begin{exercise}
Given that $f(x)=x^4-2 x^3+x^2-x-4$, evaluate $f(-0.5)$. Express your answer in decimal notation.
\begin{solution}
\begin{hint}
$f(-0.5)=(-0.5)^4-2 (-0.5)^3+(-0.5)^2-(-0.5)-4$.
\end{hint}
\begin{hint}
$f(-0.5)=-2.9375$.
\end{hint}
The value of the function $f(x) = x^4-2 x^3+x^2-x-4$, evaluated at $x=-0.5$, is $\answer{-2.9375}$.
\end{solution}
\end{exercise}

\begin{exercise}
Given that $f(x)=-2 x^3-3 x^2-4 x$, evaluate $f(2.4)$. Express your answer in decimal notation.
\begin{solution}
\begin{hint}
$f(2.4)=-2 (2.4)^3-3 (2.4)^2-4 (2.4)$.
\end{hint}
\begin{hint}
$f(2.4)=-54.528$.
\end{hint}
The value of the function $f(x) = -2 x^3-3 x^2-4 x$, evaluated at $x=2.4$, is $\answer{-54.528}$.
\end{solution}
\end{exercise}

\begin{exercise}
Given that $f(x)=-x^3-x-3$, evaluate $f(-2.9)$. Express your answer in decimal notation.
\begin{solution}
\begin{hint}
$f(-2.9)=-(-2.9)^3-(-2.9)-3$.
\end{hint}
\begin{hint}
$f(-2.9)=24.289$.
\end{hint}
The value of the function $f(x) = -x^3-x-3$, evaluated at $x=-2.9$, is $\answer{24.289}$.
\end{solution}
\end{exercise}

\begin{exercise}
Given that $f(x)=x^4+2 x^2-3 x$, evaluate $f(-1.9)$. Express your answer in decimal notation.
\begin{solution}
\begin{hint}
$f(-1.9)=(-1.9)^4+2 (-1.9)^2-3 (-1.9)$.
\end{hint}
\begin{hint}
$f(-1.9)=25.9521$.
\end{hint}
The value of the function $f(x) = x^4+2 x^2-3 x$, evaluated at $x=-1.9$, is $\answer{25.9521}$.
\end{solution}
\end{exercise}

\begin{exercise}
Given that $f(x)=x^4+x^3-2 x^2-x-2$, evaluate $f(-0.9)$. Express your answer in decimal notation.
\begin{solution}
\begin{hint}
$f(-0.9)=(-0.9)^4+(-0.9)^3-2 (-0.9)^2-(-0.9)-2$.
\end{hint}
\begin{hint}
$f(-0.9)=-2.7929$.
\end{hint}
The value of the function $f(x) = x^4+x^3-2 x^2-x-2$, evaluated at $x=-0.9$, is $\answer{-2.7929}$.
\end{solution}
\end{exercise}

\begin{exercise}
Given that $f(x)=-3 x^2-2 x+1$, evaluate $f(4.2)$. Express your answer in decimal notation.
\begin{solution}
\begin{hint}
$f(4.2)=-3 (4.2)^2-2 (4.2)+1$.
\end{hint}
\begin{hint}
$f(4.2)=-60.32$.
\end{hint}
The value of the function $f(x) = -3 x^2-2 x+1$, evaluated at $x=4.2$, is $\answer{-60.32}$.
\end{solution}
\end{exercise}

\begin{exercise}
Given that $f(x)=x^2-x-2$, evaluate $f(4.1)$. Express your answer in decimal notation.
\begin{solution}
\begin{hint}
$f(4.1)=(4.1)^2-(4.1)-2$.
\end{hint}
\begin{hint}
$f(4.1)=10.71$.
\end{hint}
The value of the function $f(x) = x^2-x-2$, evaluated at $x=4.1$, is $\answer{10.71}$.
\end{solution}
\end{exercise}

\begin{exercise}
Given that $f(x)=-2 x^2-3 x+5$, evaluate $f(-2.2)$. Express your answer in decimal notation.
\begin{solution}
\begin{hint}
$f(-2.2)=-2 (-2.2)^2-3 (-2.2)+5$.
\end{hint}
\begin{hint}
$f(-2.2)=1.92$.
\end{hint}
The value of the function $f(x) = -2 x^2-3 x+5$, evaluated at $x=-2.2$, is $\answer{1.92}$.
\end{solution}
\end{exercise}

\begin{exercise}
Given that $f(x)=-x^2+4 x-3$, evaluate $f(0.1)$. Express your answer in decimal notation.
\begin{solution}
\begin{hint}
$f(0.1)=-(0.1)^2+4 (0.1)-3$.
\end{hint}
\begin{hint}
$f(0.1)=-2.61$.
\end{hint}
The value of the function $f(x) = -x^2+4 x-3$, evaluated at $x=0.1$, is $\answer{-2.61}$.
\end{solution}
\end{exercise}

\begin{exercise}
Given that $f(x)=-x^2-4 x-2$, evaluate $f(-1.1)$. Express your answer in decimal notation.
\begin{solution}
\begin{hint}
$f(-1.1)=-(-1.1)^2-4 (-1.1)-2$.
\end{hint}
\begin{hint}
$f(-1.1)=1.19$.
\end{hint}
The value of the function $f(x) = -x^2-4 x-2$, evaluated at $x=-1.1$, is $\answer{1.19}$.
\end{solution}
\end{exercise}

\begin{exercise}
Given that $f(x)=-x^4-2 x^2-x+2$, evaluate $f(-0.3)$. Express your answer in decimal notation.
\begin{solution}
\begin{hint}
$f(-0.3)=-(-0.3)^4-2 (-0.3)^2-(-0.3)+2$.
\end{hint}
\begin{hint}
$f(-0.3)=2.1119$.
\end{hint}
The value of the function $f(x) = -x^4-2 x^2-x+2$, evaluated at $x=-0.3$, is $\answer{2.1119}$.
\end{solution}
\end{exercise}

\begin{exercise}
Given that $f(x)=5-4 x$, evaluate $f(0.5)$. Express your answer in decimal notation.
\begin{solution}
\begin{hint}
$f(0.5)=5-4 (0.5)$.
\end{hint}
\begin{hint}
$f(0.5)=3$.
\end{hint}
The value of the function $f(x) = 5-4 x$, evaluated at $x=0.5$, is $\answer{3}$.
\end{solution}
\end{exercise}

\begin{exercise}
Given that $f(x)=x^4+x-3$, evaluate $f(2.3)$. Express your answer in decimal notation.
\begin{solution}
\begin{hint}
$f(2.3)=(2.3)^4+(2.3)-3$.
\end{hint}
\begin{hint}
$f(2.3)=27.2841$.
\end{hint}
The value of the function $f(x) = x^4+x-3$, evaluated at $x=2.3$, is $\answer{27.2841}$.
\end{solution}
\end{exercise}
\end{shuffle}


\begin{shuffle}
\begin{exercise}
Given that $f(x)=\sin ^2(x)-2$, evaluate $f(\frac{5 \pi }{6})$. Express your answer in an exact form.
\begin{solution}
\begin{hint}
$f(\frac{5 \pi }{6})=\sin ^2((\frac{5 \pi }{6}))-2$. Note, you could stop here and have a perfectly acceptable answer. However, you could also recall facts about the unit circle and continue on. 
\end{hint}
\begin{hint}
$f(\frac{5 \pi }{6})=-\frac{7}{4}$.
\end{hint}
The value of the function $f(x) = \sin ^2(x)-2$, evaluated at $x=\frac{5 \pi }{6}$, is $\answer{-\frac{7}{4}}$.
\end{solution}
\end{exercise}

\begin{exercise}
Given that $f(x)=-3 \sin (2 x)+\cos ^4(2 x)-2$, evaluate $f(\frac{11 \pi }{6})$. Express your answer in an exact form.
\begin{solution}
\begin{hint}
$f(\frac{11 \pi }{6})=-3 \sin (2 (\frac{11 \pi }{6}))+\cos ^4(2 (\frac{11 \pi }{6}))-2$. Note, you could stop here and have a perfectly acceptable answer. However, you could also recall facts about the unit circle and continue on. 
\end{hint}
\begin{hint}
$f(\frac{11 \pi }{6})=\frac{3 \sqrt{3}}{2}-\frac{31}{16}$.
\end{hint}
The value of the function $f(x) = -3 \sin (2 x)+\cos ^4(2 x)-2$, evaluated at $x=\frac{11 \pi }{6}$, is $\answer{\frac{3 \sqrt{3}}{2}-\frac{31}{16}}$.
\end{solution}
\end{exercise}

\begin{exercise}
Given that $f(x)=2 \sin (x)-3$, evaluate $f(\frac{3 \pi }{2})$. Express your answer in an exact form.
\begin{solution}
\begin{hint}
$f(\frac{3 \pi }{2})=2 \sin ((\frac{3 \pi }{2}))-3$. Note, you could stop here and have a perfectly acceptable answer. However, you could also recall facts about the unit circle and continue on. 
\end{hint}
\begin{hint}
$f(\frac{3 \pi }{2})=-5$.
\end{hint}
The value of the function $f(x) = 2 \sin (x)-3$, evaluated at $x=\frac{3 \pi }{2}$, is $\answer{-5}$.
\end{solution}
\end{exercise}

\begin{exercise}
Given that $f(x)=\sin ^2(2 x)+5$, evaluate $f(\frac{7 \pi }{4})$. Express your answer in an exact form.
\begin{solution}
\begin{hint}
$f(\frac{7 \pi }{4})=\sin ^2(2 (\frac{7 \pi }{4}))+5$. Note, you could stop here and have a perfectly acceptable answer. However, you could also recall facts about the unit circle and continue on. 
\end{hint}
\begin{hint}
$f(\frac{7 \pi }{4})=6$.
\end{hint}
The value of the function $f(x) = \sin ^2(2 x)+5$, evaluated at $x=\frac{7 \pi }{4}$, is $\answer{6}$.
\end{solution}
\end{exercise}

\begin{exercise}
Given that $f(x)=-8$, evaluate $f(\frac{3 \pi }{2})$. Express your answer in an exact form.
\begin{solution}
\begin{hint}
$f(\frac{3 \pi }{2})=-8$. Note, you could stop here and have a perfectly acceptable answer. However, you could also recall facts about the unit circle and continue on. 
\end{hint}
\begin{hint}
$f(\frac{3 \pi }{2})=-8$.
\end{hint}
The value of the function $f(x) = -8$, evaluated at $x=\frac{3 \pi }{2}$, is $\answer{-8}$.
\end{solution}
\end{exercise}

\begin{exercise}
Given that $f(x)=-4$, evaluate $f(\frac{4 \pi }{3})$. Express your answer in an exact form.
\begin{solution}
\begin{hint}
$f(\frac{4 \pi }{3})=-4$. Note, you could stop here and have a perfectly acceptable answer. However, you could also recall facts about the unit circle and continue on. 
\end{hint}
\begin{hint}
$f(\frac{4 \pi }{3})=-4$.
\end{hint}
The value of the function $f(x) = -4$, evaluated at $x=\frac{4 \pi }{3}$, is $\answer{-4}$.
\end{solution}
\end{exercise}

\begin{exercise}
Given that $f(x)=\sin ^4(x)+2 \sin ^2(2 x)+1$, evaluate $f(\frac{2 \pi }{3})$. Express your answer in an exact form.
\begin{solution}
\begin{hint}
$f(\frac{2 \pi }{3})=\sin ^4((\frac{2 \pi }{3}))+2 \sin ^2(2 (\frac{2 \pi }{3}))+1$. Note, you could stop here and have a perfectly acceptable answer. However, you could also recall facts about the unit circle and continue on. 
\end{hint}
\begin{hint}
$f(\frac{2 \pi }{3})=\frac{49}{16}$.
\end{hint}
The value of the function $f(x) = \sin ^4(x)+2 \sin ^2(2 x)+1$, evaluated at $x=\frac{2 \pi }{3}$, is $\answer{\frac{49}{16}}$.
\end{solution}
\end{exercise}

\begin{exercise}
Given that $f(x)=3-2 \cos ^2(x)$, evaluate $f(\frac{5 \pi }{3})$. Express your answer in an exact form.
\begin{solution}
\begin{hint}
$f(\frac{5 \pi }{3})=3-2 \cos ^2((\frac{5 \pi }{3}))$. Note, you could stop here and have a perfectly acceptable answer. However, you could also recall facts about the unit circle and continue on. 
\end{hint}
\begin{hint}
$f(\frac{5 \pi }{3})=\frac{5}{2}$.
\end{hint}
The value of the function $f(x) = 3-2 \cos ^2(x)$, evaluated at $x=\frac{5 \pi }{3}$, is $\answer{\frac{5}{2}}$.
\end{solution}
\end{exercise}

\begin{exercise}
Given that $f(x)=0$, evaluate $f(\frac{2 \pi }{3})$. Express your answer in an exact form.
\begin{solution}
\begin{hint}
$f(\frac{2 \pi }{3})=0$. Note, you could stop here and have a perfectly acceptable answer. However, you could also recall facts about the unit circle and continue on. 
\end{hint}
\begin{hint}
$f(\frac{2 \pi }{3})=0$.
\end{hint}
The value of the function $f(x) = 0$, evaluated at $x=\frac{2 \pi }{3}$, is $\answer{0}$.
\end{solution}
\end{exercise}

\begin{exercise}
Given that $f(x)=-\cos (x)-4$, evaluate $f(\frac{11 \pi }{6})$. Express your answer in an exact form.
\begin{solution}
\begin{hint}
$f(\frac{11 \pi }{6})=-\cos ((\frac{11 \pi }{6}))-4$. Note, you could stop here and have a perfectly acceptable answer. However, you could also recall facts about the unit circle and continue on. 
\end{hint}
\begin{hint}
$f(\frac{11 \pi }{6})=-4-\frac{\sqrt{3}}{2}$.
\end{hint}
The value of the function $f(x) = -\cos (x)-4$, evaluated at $x=\frac{11 \pi }{6}$, is $\answer{-4-\frac{\sqrt{3}}{2}}$.
\end{solution}
\end{exercise}

\begin{exercise}
Given that $f(x)=-4$, evaluate $f(\frac{\pi }{4})$. Express your answer in an exact form.
\begin{solution}
\begin{hint}
$f(\frac{\pi }{4})=-4$. Note, you could stop here and have a perfectly acceptable answer. However, you could also recall facts about the unit circle and continue on. 
\end{hint}
\begin{hint}
$f(\frac{\pi }{4})=-4$.
\end{hint}
The value of the function $f(x) = -4$, evaluated at $x=\frac{\pi }{4}$, is $\answer{-4}$.
\end{solution}
\end{exercise}

\begin{exercise}
Given that $f(x)=-\sin ^3(2 x)-3$, evaluate $f(\frac{5 \pi }{4})$. Express your answer in an exact form.
\begin{solution}
\begin{hint}
$f(\frac{5 \pi }{4})=-\sin ^3(2 (\frac{5 \pi }{4}))-3$. Note, you could stop here and have a perfectly acceptable answer. However, you could also recall facts about the unit circle and continue on. 
\end{hint}
\begin{hint}
$f(\frac{5 \pi }{4})=-4$.
\end{hint}
The value of the function $f(x) = -\sin ^3(2 x)-3$, evaluated at $x=\frac{5 \pi }{4}$, is $\answer{-4}$.
\end{solution}
\end{exercise}

\begin{exercise}
Given that $f(x)=1$, evaluate $f(\frac{7 \pi }{6})$. Express your answer in an exact form.
\begin{solution}
\begin{hint}
$f(\frac{7 \pi }{6})=1$. Note, you could stop here and have a perfectly acceptable answer. However, you could also recall facts about the unit circle and continue on. 
\end{hint}
\begin{hint}
$f(\frac{7 \pi }{6})=1$.
\end{hint}
The value of the function $f(x) = 1$, evaluated at $x=\frac{7 \pi }{6}$, is $\answer{1}$.
\end{solution}
\end{exercise}

\begin{exercise}
Given that $f(x)=\sin ^4(2 x)-2 \cos ^3(x)+5$, evaluate $f(\frac{7 \pi }{4})$. Express your answer in an exact form.
\begin{solution}
\begin{hint}
$f(\frac{7 \pi }{4})=\sin ^4(2 (\frac{7 \pi }{4}))-2 \cos ^3((\frac{7 \pi }{4}))+5$. Note, you could stop here and have a perfectly acceptable answer. However, you could also recall facts about the unit circle and continue on. 
\end{hint}
\begin{hint}
$f(\frac{7 \pi }{4})=6-\frac{1}{\sqrt{2}}$.
\end{hint}
The value of the function $f(x) = \sin ^4(2 x)-2 \cos ^3(x)+5$, evaluated at $x=\frac{7 \pi }{4}$, is $\answer{6-\frac{1}{\sqrt{2}}}$.
\end{solution}
\end{exercise}

\begin{exercise}
Given that $f(x)=-3 \sin ^2(2 x)+4 \sin (x)-5$, evaluate $f(\frac{5 \pi }{4})$. Express your answer in an exact form.
\begin{solution}
\begin{hint}
$f(\frac{5 \pi }{4})=-3 \sin ^2(2 (\frac{5 \pi }{4}))+4 \sin ((\frac{5 \pi }{4}))-5$. Note, you could stop here and have a perfectly acceptable answer. However, you could also recall facts about the unit circle and continue on. 
\end{hint}
\begin{hint}
$f(\frac{5 \pi }{4})=-8-2 \sqrt{2}$.
\end{hint}
The value of the function $f(x) = -3 \sin ^2(2 x)+4 \sin (x)-5$, evaluated at $x=\frac{5 \pi }{4}$, is $\answer{-8-2 \sqrt{2}}$.
\end{solution}
\end{exercise}

\begin{exercise}
Given that $f(x)=0$, evaluate $f(\frac{11 \pi }{6})$. Express your answer in an exact form.
\begin{solution}
\begin{hint}
$f(\frac{11 \pi }{6})=0$. Note, you could stop here and have a perfectly acceptable answer. However, you could also recall facts about the unit circle and continue on. 
\end{hint}
\begin{hint}
$f(\frac{11 \pi }{6})=0$.
\end{hint}
The value of the function $f(x) = 0$, evaluated at $x=\frac{11 \pi }{6}$, is $\answer{0}$.
\end{solution}
\end{exercise}

\begin{exercise}
Given that $f(x)=4 \cos (2 x)-4$, evaluate $f(\frac{5 \pi }{3})$. Express your answer in an exact form.
\begin{solution}
\begin{hint}
$f(\frac{5 \pi }{3})=4 \cos (2 (\frac{5 \pi }{3}))-4$. Note, you could stop here and have a perfectly acceptable answer. However, you could also recall facts about the unit circle and continue on. 
\end{hint}
\begin{hint}
$f(\frac{5 \pi }{3})=-6$.
\end{hint}
The value of the function $f(x) = 4 \cos (2 x)-4$, evaluated at $x=\frac{5 \pi }{3}$, is $\answer{-6}$.
\end{solution}
\end{exercise}

\begin{exercise}
Given that $f(x)=\sin ^4(2 x)+\sin ^3(2 x)-4 \sin (2 x)-1$, evaluate $f(\pi)$. Express your answer in an exact form.
\begin{solution}
\begin{hint}
$f(\pi)=\sin ^4(2 (\pi))+\sin ^3(2 (\pi))-4 \sin (2 (\pi))-1$. Note, you could stop here and have a perfectly acceptable answer. However, you could also recall facts about the unit circle and continue on. 
\end{hint}
\begin{hint}
$f(\pi)=-1$.
\end{hint}
The value of the function $f(x) = \sin ^4(2 x)+\sin ^3(2 x)-4 \sin (2 x)-1$, evaluated at $x=\pi$, is $\answer{-1}$.
\end{solution}
\end{exercise}

\begin{exercise}
Given that $f(x)=2-2 \cos (x)$, evaluate $f(\pi)$. Express your answer in an exact form.
\begin{solution}
\begin{hint}
$f(\pi)=2-2 \cos ((\pi))$. Note, you could stop here and have a perfectly acceptable answer. However, you could also recall facts about the unit circle and continue on. 
\end{hint}
\begin{hint}
$f(\pi)=4$.
\end{hint}
The value of the function $f(x) = 2-2 \cos (x)$, evaluated at $x=\pi$, is $\answer{4}$.
\end{solution}
\end{exercise}

\begin{exercise}
Given that $f(x)=2 \cos ^2(2 x)+5$, evaluate $f(\frac{7 \pi }{4})$. Express your answer in an exact form.
\begin{solution}
\begin{hint}
$f(\frac{7 \pi }{4})=2 \cos ^2(2 (\frac{7 \pi }{4}))+5$. Note, you could stop here and have a perfectly acceptable answer. However, you could also recall facts about the unit circle and continue on. 
\end{hint}
\begin{hint}
$f(\frac{7 \pi }{4})=5$.
\end{hint}
The value of the function $f(x) = 2 \cos ^2(2 x)+5$, evaluated at $x=\frac{7 \pi }{4}$, is $\answer{5}$.
\end{solution}
\end{exercise}
\end{shuffle}
