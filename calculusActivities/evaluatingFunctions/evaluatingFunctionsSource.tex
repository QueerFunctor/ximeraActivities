\headline{In this activity we practice evaluating functions at numbers and other functions.} % A one sentence description of the activity
\activitytitle{Evaluating Functions} % the title of the activity

We'll start off easy, asking you a set of questions that you can
probably do.


\begin{shuffle}
\begin{exercise}
Given that $f(x)=-5 x^4+2 x^3+x^2-3 x+2$, evaluate $f(3.9)$.
\begin{solution}
\begin{hint}
$f(3.9)=-5 (3.9)^4+2 (3.9)^3+(3.9)^2-3 (3.9)+2$.
\end{hint}
\begin{hint}
$f(3.9)=-1032.57$.
\end{hint}
The value of the function $f(x) = -5 x^4+2 x^3+x^2-3 x+2$, evaluated at $x=3.9$, is $\answer{-1032.57}$.
\end{solution}
\end{exercise}

\begin{exercise}
Given that $f(x)=-2 x^4-2 x^3-3 x^2+3 x+4$, evaluate $f(-0.6)$.
\begin{solution}
\begin{hint}
$f(-0.6)=-2 (-0.6)^4-2 (-0.6)^3-3 (-0.6)^2+3 (-0.6)+4$.
\end{hint}
\begin{hint}
$f(-0.6)=1.2928$.
\end{hint}
The value of the function $f(x) = -2 x^4-2 x^3-3 x^2+3 x+4$, evaluated at $x=-0.6$, is $\answer{1.2928}$.
\end{solution}
\end{exercise}

\begin{exercise}
Given that $f(x)=-5 x^3-5 x^2+5 x-1$, evaluate $f(4.5)$.
\begin{solution}
\begin{hint}
$f(4.5)=-5 (4.5)^3-5 (4.5)^2+5 (4.5)-1$.
\end{hint}
\begin{hint}
$f(4.5)=-535.375$.
\end{hint}
The value of the function $f(x) = -5 x^3-5 x^2+5 x-1$, evaluated at $x=4.5$, is $\answer{-535.375}$.
\end{solution}
\end{exercise}

\begin{exercise}
Given that $f(x)=-3 x^4-2 x^3-5 x^2+x+5$, evaluate $f(-4.2)$.
\begin{solution}
\begin{hint}
$f(-4.2)=-3 (-4.2)^4-2 (-4.2)^3-5 (-4.2)^2+(-4.2)+5$.
\end{hint}
\begin{hint}
$f(-4.2)=-872.733$.
\end{hint}
The value of the function $f(x) = -3 x^4-2 x^3-5 x^2+x+5$, evaluated at $x=-4.2$, is $\answer{-872.733}$.
\end{solution}
\end{exercise}
\end{shuffle}



\begin{exercise}
Given that $f(x)=-5 \sin ^2(2 x)-1$, evaluate $f(2 \pi)$.
\begin{solution}
\begin{hint}
$f(2 \pi)=-5 \sin ^2(2 (2 \pi))-1$.
\end{hint}
\begin{hint}
$f(2 \pi)=-1$.
\end{hint}
The value of the function $f(x) = -5 \sin ^2(2 x)-1$, evaluated at \
$x=2 \pi$, is $\answer{-1}$.
\end{solution}
\end{exercise}


\begin{exercise}
A lone exercise.
\end{exercise}

\begin{exercise}
A second lone exercise.
\end{exercise}


\begin{shuffle}
\begin{question}
A not so lone exercise.
\end{question}

\begin{question}
Another not so lone exercise.
\end{question}

\end{shuffle}
